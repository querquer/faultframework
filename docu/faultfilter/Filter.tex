\documentclass[a4paper,11pt]{article}
\usepackage[T1]{fontenc}
\usepackage[utf8]{inputenc}
\usepackage{lmodern}
\usepackage[english]{babel}

\title{List of sensor fault filter methods}
\author{Jonas Marquardt}

\begin{document}

\maketitle
\tableofcontents

\begin{abstract}
 This document is meant to list up different methods for sensor fault filtration. Furthermore, all methods should be classified whether they can handle specific fault types or not. As we assume the sensor signal as the only input of the fault detection methods, another classification is done concerning the dynamic of the input signal. This classification is done by deciding whether a method is able to filter a specific fault type in a high/middle/low dynamic signal. 
\end{abstract}

\section{Preparation}
 
	\subsection{Fault model}  

		%TODO Fault model von Sebastian erklären? ... reicht eigentlich einmal in gesamt-Doku

	\subsection{Process model}

\section{Filter classification}

\begin{itemize}
\item[FIR-Filter]
%	\subitem[sind Stabie]
%	\subitem[endlich lange Impulsantwort]

	\subitem CIC-Filter
	\subitem Medianfilter
%		\subsubitem nichtlinearer filter
%		\subsubitem aus den werten in der umgebung wird der wert genommen, der nach einer sortierung in der mitte steht.

	\subitem Mittelwertfilter / gleitender Mittelwert
%		\subsubitem[aufsummieren aufeinander folgender werte, division durch denren Anzahl.]

	\subitem Gleitende gewichtete Mittelwerte
%		\subsubitem -> wie mittelwert nur mit wichtung der eingehenden werte

	\subitem Reduktionsfilter
%		\subsubitem Mehrere Punkte werden zu einem zusammengefasst

	\subitem Binomialfilter/Gausfilter

\item[IIR-Filter]
%	\subitem unendlich lange impulsantwort
%	\subitem sind rekursiv

	\subitem Wellendigitalfilter

\item[Latticefilter]
%	\subitem können als IIR oder FIR ausgeführt werden.

\item[Kaskadierte IIR-Filter]

\item[adaptive Filter]

\item[Transversalfilter]

\item[Kalman Filter]
%		\subitem Vorhersageschritt + Korrekturschritt

\item[Bayesscher Filter]
%		\subitem wenn wert so..., wie wahrscheinlich das es so... ist?

\item[Multiratenfilter]

\item[Frequenzverzerrte Filter]

\item[wraped FIR]

\item[wraped IIR]
\end{itemize}
		

%Sensor fault detection was addressed by many scientific publications and papers before. Therefore a lot of different approaches available. As the primary focus is on the framework, we only analyze some data-driven techniques on detecting sensor faults. This means, the input of every detection method is a signal representing the observations of an sensor over time. Beside this, we assume only one-dimensional sensors, e.g distance sensors. As the database will be of the same dimension as the input of the methods, the dimension will be 2. 

%Furthermore, we want to provide a kind of categorization of this data-driven methods: 
%There are three different types of data-driven methods. At first, the fault detection techniques based on pattern recognition. This type of methods analyze the input data in order to find specified patterns, such as fault patterns. Prominent examples of this category are neural networks like \textit{Time-Delay neural networks} and also classifications-methods like nearest-neighbor-classification.

%Another type of detection techniques are using time-redundancy for residual or symptom generation. This methods tries to find redundant information in a time-series, such as the average which must be in a certain interval. If this information is deviating from normal values, sensor faults are detected. Examples of this category are \textit{Limit Checking} on the gradient of a signal or an auto-correlation-analysis.

%The last type are using models of the underlaying process. By predicting and comparing the sensor observations, they can generate a residual. The process model can be obtained by different approaches. As an example, one could use recurrent neural networks (Elman-networks, NARX) in order to \glqq learn \grqq the model. Another possibility is to determine the transfer function of the system.

%\subsection{Recognizing patterns of sensor faults}

%\subsection{Using time-redundancy for residual/symptom generation}


% \begin{table}[h]
% \caption{Classification: \glqq Limit-checking of the signals gradient \grqq}
% \begin{tabular}{c|c|c|c}
%                    & High dynamic        & Normal dynamic        & Low dynamic \\  \hline
% 1. Outlier         & not appropriate     &      OK               &       Well
% \end{tabular} 
% \label{tab:lc_gradient}
% \end{table}


%\subsection{Using process models for residual/symptom generation}

\end{document}
