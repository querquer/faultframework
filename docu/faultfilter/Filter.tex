\documentclass[a4paper,11pt]{article}
\usepackage[T1]{fontenc}
\usepackage[utf8]{inputenc}
\usepackage{lmodern}
\usepackage[english]{babel}

\title{List of sensor fault filter methods}
\author{Jonas Marquardt}

\begin{document}

\maketitle
\tableofcontents

\begin{abstract}
 This document is meant to list up different methods for sensor fault filtration. Furthermore, all methods should be classified whether they can handle specific fault types or not. As we assume the sensor signal as the only input of the fault detection methods, another classification is done concerning the dynamic of the input signal. This classification is done by deciding whether a method is able to filter a specific fault type in a high/middle/low dynamic signal. 
\end{abstract}

\section{Preparation}
 
	\subsection{Fault model}  

		%TODO Fault model von Sebastian erklären? ... reicht eigentlich einmal in gesamt-Doku

	\subsection{Process model}

\section{Filter classification}

\begin{itemize}
\item[FIR-Filter]
%	\subitem[sind Stabie]
%	\subitem[endlich lange Impulsantwort]

	\subitem CIC-Filter
	\subitem wraped FIR
	\subitem Medianfilter (umgesetzt)
%		\subsubitem nichtlinearer filter
%		\subsubitem aus den werten in der umgebung wird der wert genommen, der nach einer sortierung in der mitte steht.

	\subitem Mittelwertfilter / gleitender Mittelwert
%		\subsubitem[aufsummieren aufeinander folgender werte, division durch denren Anzahl.]

	\subitem Gleitende gewichtete Mittelwerte
%		\subsubitem -> wie mittelwert nur mit wichtung der eingehenden werte

	\subitem Reduktionsfilter
%		\subsubitem Mehrere Punkte werden zu einem zusammengefasst

	\subitem Binomialfilter/Gausfilter

\item[IIR-Filter/Transversalfilter]
%	\subitem unendlich lange impulsantwort
%	\subitem sind rekursiv

	\subitem wraped IIR
	\subitem Wellendigitalfilter
	\subitem Kaskadierte IIR-Filter

\item[Latticefilter]
%	\subitem können als IIR oder FIR ausgeführt werden.


\item[Kalman Filter]
%		\subitem Vorhersageschritt + Korrekturschritt

\item[Bayesscher Filter]
%		\subitem wenn wert so..., wie wahrscheinlich das es so... ist?

\end{itemize}
		

\end{document}
